\documentclass[9pt]{beamer} 
\usetheme{cwc} 

\usepackage{amsmath,amssymb}
\usepackage{graphicx,acronym,setspace,epstopdf}
\usepackage[ruled]{algorithm2e}
\usepackage{subeqnarray,multirow,cite,array,color,mathtools,setspace,geometry}

\acrodef{MSE}{mean squared error}
\acrodef{BC}{broadcast channel}
\acrodef{MC}{multi-cell}
\acrodef{BS}{base station}
\acrodef{MIMO}{multiple-input multiple-output}
\acrodef{SISO}{single-input single-output}
\acrodef{MU}{multi-user}
\acrodef{MU-MIMO}{\acl{MU} \acl{MIMO}}
\acrodef{OFDM}{orthogonal frequency division multiplexing}
\acrodef{WSRM}{weighted sum rate maximization}
\acrodef{QoS}{quality of service}
\acrodef{SCA}{successive convex approximation}
\acrodef{SNR}{signal-to-noise ratio}
\acrodef{MMSE}{minimum \acl{MSE}}
\acrodef{SIR}{signal-to-interference ratio}
\acrodef{SINR}{signal-to-interference-plus-noise ratio}
\acrodef{Q-WSRM}{queue \acl{WSRM}}
\acrodef{QM}{queue minimizing}
\acrodef{SRA}{spatial resource allocation}
\acrodef{JSFRA}{joint space-frequency resource allocation}
\acrodef{WMMSE}{weighted \acl{MMSE}}
\acrodef{KKT}{Karush-Kuhn-Tucker}
\acrodef{GP}{geometric programming}
\acrodef{SOC}{second-order cone}
\acrodef{BCDM}{block coordinate descent method}

\newcommand{\mbf}[1]{\mathbf{#1}}
\newcommand{\me}[1]{\( #1 \)}
\newcommand{\mc}[1]{\mathcal{#1}}
\newcommand{\fall}{\forall}
\newcommand{\set}[1]{\left \lbrace #1 \right \rbrace }
\newcommand{\mvec}[2]{\mathbf{#1}_{#2}}
\newcommand{\ith}[1]{{#1}^\mathrm{th}}
\newcommand{\pr}[1]{{#1}^\prime}
\newcommand{\mbfa}[1]{{\boldsymbol{#1}}}
\newcommand{\herm}{\mathrm{H}}
\newcommand{\sset}[1]{\left [ #1 \right ]}
\newcommand{\rfrac}[2]{{}^{#1}/{}_{#2}}
\newcommand{\eqspace}{\IEEEeqnarraynumspace}
\newcommand{\enoise}{\widetilde{N}_0}
\newcommand{\eqsub}{\IEEEyessubnumber}
\newcommand{\review}[1]{{\textcolor[rgb]{0 0 0.6}{#1}}}
\newcommand{\trace}{\mathrm{tr}}
\newcommand{\tran}{\mathrm{T}}
\newcommand{\R}[1]{\label{#1}\linelabel{#1}}
\newcommand{\lr}[1]{page~\pageref{#1}, line~\lineref{#1}}
\newcommand{\eqn}[1]{\(#1\)}
\newcommand{\mx}{\mbf{m}}
\newcommand{\my}{\mbf{w}}
\newcommand{\mz}{\mbfa{\gamma}}
\newcommand{\mxb}{{{\mbf{m}}}}
\newcommand{\myb}{{{\mbf{w}}}}
\newcommand{\iterate}[2]{{#1}^{(#2)}}
\newcommand{\iter}[3]{{#1}_{#2}^{(#3)}}
\newcommand{\ma}{\mbf{x}}
\acrodef{MSE}{mean squared error}
\acrodef{IBC}{interference broadcast channel}
\acrodef{MC}{multi-cell}
\acrodef{BS}{base station}
\acrodef{MIMO}{multiple-input multiple-output}
\acrodef{SISO}{single-input single-output}
\acrodef{MU}{multiple users}
\acrodef{OFDM}{orthogonal frequency division multiplexing}
\acrodef{WSRM}{weighted sum rate maximization}
\acrodef{QoS}{quality of service}
\acrodef{SCA}{successive convex approximation}
\acrodef{SNR}{signal-to-noise ratio}
\acrodef{MMSE}{minimum \acl{MSE}}
\acrodef{SIR}{signal-to-interference ratio}
\acrodef{SINR}{signal-to-interference-plus-noise ratio}
\acrodef{Q-WSRM}{queue \acl{WSRM}}
\acrodef{QM}{queue minimizing}
\acrodef{SRA}{spatial resource allocation}
\acrodef{JSFRA}{joint space-frequency resource allocation}
\acrodef{WMMSE}{weighted \acl{MMSE}}
\acrodef{KKT}{Karush-Kuhn-Tucker}
\acrodef{GP}{geometric programming}
\acrodef{SOC}{second-order cone}
%\acrodef{BCDM}{block coordinate descent method}
\acrodef{ADMM}{alternating directions method of multipliers}
\acrodef{PD}{primal decomposition}
\acrodef{DD}{dual decomposition}
\acrodef{FFR}{fractional frequency reuse}
\acrodef{DC}{difference of convex}
\acrodef{Q-WSRME}{\ac{Q-WSRM} extended}
\acrodef{TDD}{time division duplexing}
\acrodef{CSI}{channel state information}
\acrodef{AO}{alternating optimization}
\acrodef{OTA}{over-the-air}
\acrodef{PL}{path loss}
\acrodef{TDM}{time division multiplexing}
\acrodef{UC}{uncoordinated}
\acrodef{SoC}{system-on-chip}
\acrodef{WMMSE}{weighted minimum mean squared error}
\acrodef{AMBA}{Advanced Microcontroller Bus Architecture}
\acrodef{AXI}{Advanced Extensible Interface}
\acrodef{FPGA}{Field-programmable gate array}
\acrodef{MSMCSRAM}{multi-core shared memory}
\acrodef{IPC}{inter-processor communications}
\acrodef{CIC}{chip-level interrupt controller}
\acrodef{MCSDK}{multi-core software development kit}
\acrodef{AMP}{Asymmetric Multi-Processing}

\graphicspath{{./../Figures/}{./../Figures/Linux/}{./../Figures/Review/}}
\DeclareGraphicsExtensions{.eps}

\epstopdfsetup{update,prepend,prefersuffix=false,suffix=}
\DeclareGraphicsRule{.eps}{pdf}{.pdf}{`epstopdf #1}
\pdfcompresslevel=9

\linespread{1.4}

\title{Traffic Aware Precoder Design for Space Frequency Resource Allocation}
\author{{Ganesh Venkatraman\eqn{^\dagger}, Antti T\"{o}lli, Le-Nam Tran and Markku Juntti} \\ \scriptsize{Email: \{gvenkatr, antti.tolli, le.nam.tran, markku.juntti\}@ee.oulu.fi}}


\begin{document}

\AtBeginSection{\frame{\sectionpage}}

\begin{frame}
    \titlepage
\end{frame}

\begin{frame}{Outline} \scriptsize
    \tableofcontents
\end{frame}

\acused{BS} \acused{MIMO} \acused{OFDM}

\section{Introduction}

\begin{frame}{Introduction and Motivation}
\begin{itemize}
\item Complex algorithms are adopted to maximize throughput to satisfy the data requirements from higher layers
\item Available wireless resources are to be utilized efficiently to minimize the backlogged packets 
\item Spatial and Frequency resources are exploited to empty the packets waiting at the \acp{BS}
\item We discuss precoder designs for \acl{MU} \acs{MIMO}-\acs{OFDM} setup to minimize the number of queued packets 
\end{itemize}
\end{frame}

\section{System Model \& Problem Formulation}

\begin{frame}{Symbols used}
\begin{itemize}
\item \acs{OFDM} system with \me{N} sub-channels and \me{N_B} \acp{BS}, each equipped with \me{N_T} transmit antennas
\item Let \me{K} be the total number of users with \me{N_R} antennas
\item Let \me{\mc{B}} and \me{\mc{U}} denote the set of coordinating \acp{BS} and users in the system
\item Let \me{L} be the total available spatial streams for a user \me{k}, given by \me{\min (N_T,N_R)}
\end{itemize}
\end{frame}

\begin{frame}{System Model}
\begin{itemize}
\item The \me{\ith{l}} spatial signal received on sub-channel \me{n} of user \me{k} is given by
\begin{multline}\label{eqn-1}
\hat{d}_{l,k,n} = \mvec{w}{l,k,n}^\herm \mvec{H}{b_k,k,n} \,\mvec{m}{l,k,n} d_{l,k,n} + \mvec{w}{l,k,n}^\herm \mvec{n}{l,k,n} \\ 
+ \mvec{w}{l,k,n}^\herm \sum_{i \in \mc{U} \backslash \set{k}} \mvec{H}{b_i,k,n} \sum_{j = 1}^L \mvec{m}{j,i,n}d_{j,i,n}
\end{multline}
\item where \me{\mvec{m}{l,k,n}} and \me{\mvec{w}{l,k,n}} are transmit and receive beamformers corresponding to the \me{\ith{l}} spatial stream on the \me{\ith{n}} sub-channel of user \me{k}
\end{itemize}
\end{frame}

\begin{frame}{System Model}
\begin{itemize}
\item \me{\mvec{H}{b_k,k,n} \in \mathbb{C}^{N_R \times N_T}} denotes the channel between \ac{BS} \me{b_k} and user \me{k}
\item \me{d_{l,k,n}} and \me{{n}_{l,k,n}} correspond to data symbol and equivalent noise on \me{\ith{l}} spatial stream of user \me{k}
\item Using the above notations, the \acs{SINR} seen by the \me{\ith{l}} spatial stream on the \me{\ith{n}} sub-channel for user \me{k} is given by
\end{itemize}
\begin{equation}\label{eq:SINR}
\gamma_{l,k,n} = \dfrac{\left |\mvec{w}{l,k,n}^\herm \, \mvec{H}{b_k,k,n} \, \mvec{m}{l,k,n} \right |^2}{\enoise + \sum_{(j,i) \neq (l,k)} |\mvec{w}{l,k,n}^\herm \mvec{H}{b_i,k,n} \mvec{m}{j,i,n} |^2}
\end{equation}
\begin{itemize}
\item where \eqn{\enoise = \|\mvec{w}{l,k,n}^\herm \mvec{n}{l,k,n} \|^2 }
\end{itemize}
\end{frame}

\begin{frame}{Queueing Model}
\begin{itemize}
\item Each user is associated with backlogged packets of size \me{Q_k} packets.
\item Queued packets \me{Q_k} of each user follows dynamic equation at the \me{\ith{i}} instant as
\begin{equation}
Q_k(i+1) = \Big [ Q_k(i) - t_k(i) \Big ]^+ + \lambda_k(i)
\label{eqn-2a}
\end{equation}
\item where \me{t_k = \sum_{n = 1}^N \, \sum_{l = 1}^L \, t_{l,k,n}} denotes the total number of transmitted packets corresponding to user \me{k} in the previous \me{\ith{i}} instant
\item \me{\lambda_k} represents the fresh arrivals of user \me{k} at \ac{BS} \me{b_k}
\end{itemize}
\end{frame}

\begin{frame}{Problem Formulation}
\begin{itemize}
\item {\color{red}Objective} - to minimize the number of backlogged packets waiting at \acp{BS}
\item {\color{red}Optimization variables} - transmit precoders and receive beamformers
\item {\color{red}\ac{MIMO}-\ac{OFDM}} - scheduling of users across sub-channels is inherently performed by precoders
\end{itemize}
\end{frame}

\section{Centralized Solutions}

\subsection{Existing \acs{Q-WSRM} Formulation}

\begin{frame}{Queue-Weighted Sum Rate Maximization (\acs{Q-WSRM})}
\begin{itemize}
\item \acs{Q-WSRM} formulation is the result of \textcolor{blue}{minimizing the conditional Lyapunov drift}\eqn{^\dagger}
\item \acs{Q-WSRM} formulation is also called as \alert{back pressure algorithm}, since it acts greedily in minimizing the backlogged packets at each instant
\[ \underset{t_{l,k,n}}{\text{minimize}} \quad \sum_{k \in \mc{U}} \left \lbrace Q_k(i)^2 - Q_k(i-1)^2 \right \rbrace, \]
\item where \me{Q_k} follows the dynamic Queue expression in \eqref{eqn-2a}
\end{itemize}

\vspace{2eM}
\eqn{^\dagger}\footnotesize{Neely, Michael J. "Stochastic network optimization with application to communication and queueing systems." Synthesis Lectures on Communication Networks 3.1 (2010): 1-211.}
\end{frame}

\begin{frame}{Queue-Weighted Sum Rate Maximization (\acs{Q-WSRM})}
\begin{itemize}
\item \acs{Q-WSRM} formulation, which is obtained by solving Lyapunov drift, is given by
\end{itemize}
\begin{subequations}
\begin{align}
 \underset{t_{l,k,n}}{\text{maximize}} & \qquad \sum_{k \in \mc{U}} \; Q_k \left ( \alert{\sum_{n=1}^N} \, \sum_{l = 1}^L  t_{l,k,n} \right ) \\
& \qquad {\color{blue} \alert{\sum_{n=1}^N} \, \sum_{l = 1}^L  t_{l,k,n}  \leq \alert{Q_k} \; / \; Q_{k,n}}
\end{align}
\end{subequations}
\begin{itemize}
\item Queue-Rate product is maximized
\item Users with more number of backlogged packets are favored over good channel users
\end{itemize}
\end{frame}

\subsection{\acs{JSFRA} Formulation (\acs{SINR} Relaxation)}

\begin{frame}{\acs{JSFRA} Formulation (\acs{SINR} Relaxation)}
\begin{itemize}
\item Centralized Design - precoders are designed by a controller, which are then used by all \acp{BS} in \me{\mc{B}}
\item The objective used to design transmit precoders is 
\begin{equation}
\scriptsize {\color{blue} v_k = \left | Q_k - \sum_{n = 1}^N \sum_{l = 1}^{L} t_{l,k,n} \right |^q }
\end{equation}
\item To generalize the objective, we use \me{\tilde{v}_k \triangleq a_k^{\frac{1}{q}} \, v_k}, where \me{a_k} is arbitrary weights used control the priorities
\item Exponent \me{q} plays different role based on the value it assumes
	\begin{itemize}
	\item \me{\ell_{q=1}} \alert{results in greedy allocation}
	\item \me{\ell_{q=2}} \alert{ideal for the delay or buffer size limited scenarios}
	\item \me{\ell_{q=\infty}} \alert{provides fair resource allocation in each transmission instant}
	\end{itemize}
\end{itemize}
\end{frame}

\begin{frame}{\acs{JSFRA} Formulation (\acs{SINR} Relaxation)}
\begin{itemize}
\item Optimization problem with queue difference objective is nonconvex due to the constraint
\begin{subequations}  \scriptsize
\begin{align}
\underset{\mvec{m}{l,k,n}}{\text{minimize}} & \qquad \left | Q_k - \sum_{n = 1}^N \sum_{l = 1}^{L} \log \left (1 + \gamma_{l,k,n} \right) \right |^q \\
\text{subject to} & \qquad \alert{\gamma_{l,k,n} \leq \dfrac{\left |\mvec{w}{l,k,n}^H \, \mvec{H}{b_k,k,n} \, \mvec{m}{l,k,n} \right |}{\beta_{l,k,n}}^2 } \\
& \qquad \enoise + \sum_{(j,i) \neq (l,k)} |\mvec{w}{l,k,n}^\herm \mvec{H}{b_i,k,n} \mvec{m}{j,i,n} |^2 \leq \beta_{l,k,n}
\end{align}
\end{subequations}
\item The nonconvex constraints are approximated by sequence of convex subsets and solved iteratively by \alert{\ac{SCA} method}
\item Receive beamformers are designed by the \acs{MMSE} receivers using the converged transmit precoders
\end{itemize}
\end{frame}

\subsection{\acs{JSFRA} Formulation (\acs{MSE} Reformulation)}

\begin{frame}{\acs{JSFRA} Formulation (\acs{MSE} Reformulation)}
	\begin{itemize}
	\item Alternatively, we solve the queue minimization problem by utilizing the relation between the \acs{MSE} and the \acs{SINR} as 
	\begin{eqnarray}
	\color{blue} \epsilon_{l,k,n} = (1 + \gamma_{l,k,n})^{-1}
	\end{eqnarray}
	\item Equivalence is valid only when the receivers are designed with the \ac{MSE} objective, \textit{i.e.}, \textcolor{blue}{using \acs{MMSE} receivers}
	\item Problem involves nonconvex constraint 
	\begin{subeqnarray}
	\color{red}{t_{l,k,n}} &\alert{\leq}& \alert{-\log_2(\epsilon_{l,k,n})} \\
	\epsilon_{l,k,n} &=& \mathbb{E} \big [ ( d_{l,k,n} - \hat{d}_{l,k,n} )^2 \big ] = \big | 1 - \mvec{w}{l,k,n}^\herm \mvec{H}{b_k,k,n} \mvec{m}{l,k,n} \big |^2 \nonumber \\
	&\quad& + \sum_{{(j,i) \neq (l,k)}} \big | \mvec{w}{l,k,n}^\herm \mvec{H}{b_i,k,n} \mvec{m}{j,i,n} \big |^2 + \enoise
	\end{subeqnarray}
	\end{itemize}
\end{frame}

\begin{frame}{\acs{JSFRA} Formulation (\acs{MSE} Reformulation)}
	\begin{itemize}
		\item The nonconvex constraint is approximated by a sequence of convex constraints, which is performed using \ac{SCA} technique
		\item The iterative procedure is carried out until convergence or for suitable number of iterations
		\item \alert{The above reformulation works only with the \acs{MMSE} receiver}
	\end{itemize}
\end{frame}

\section{Distributed Solutions}

\subsection{Primal \& \acs{ADMM} based decompositions}

\begin{frame}{Distributed Methods}
	\begin{itemize}
		\item \alert{Small System - centralized approach is viable}, provided channel remains constant for multiple transmission slots
		\item However, overhead involved in the centralized design scales up significantly as the network size grows
		\item Distributed approaches based on primal decomposition or \acs{ADMM} can be used to reduce the signaling requirements
		\item Signaling involved in the design of precoders are only \alert{scalar interference variables}
		\item Approximated convex subproblem in each \acs{SCA} step is performed via distributed methods
	\end{itemize}
\end{frame}

\begin{frame}{Primal Decomposition Method}
	\begin{itemize}
		\item Precoder design is based on master-slave approach
		\item Interference to the neighboring \ac{BS} users are \alert{bounded by a scalar variable, treated as a constant in subproblems}
		\item The interference thresholds are determined by the master problem and used in each slave subproblem constraint as
		\begin{equation} \label{inter_exp} \scriptsize
		\zeta_{l,k,n,b} \geq \sum_{i \in \mc{U}_b} \sum_{j = 1}^L |\mvec{w}{l,k,n}^\herm \mvec{H}{b,k,n} \mvec{m}{j,i,n} |^2 \; \forall b \in \bar{\mc{B}}_{b_k}.
		\end{equation}
	\end{itemize}
\end{frame}

\acused{ADMM}
\begin{frame}{ADMM based Decomposition Method}
	\begin{itemize}
		\item The \ac{ADMM} is superior to other distributed schemes in terms of the convergence speed
		\item \ac{ADMM} includes an additional quadratic term 
		\begin{equation}
		\color{blue} \|v_k\|_q + \sum_{l,k,n} \nu_{l,k,n}^{(j)} \left ( \mbfa{\zeta}_b - \mbfa{\zeta}^{(j)}_b \right ) + \frac{\rho}{2}  \| \mbfa{\zeta}_b - \mbfa{\zeta}^{(j)}_b \|^2
		\end{equation}
		in objective, where \eqn{\mbfa{\zeta}^{(j)}_b} is global consensus variable
		\item \eqn{\zeta_{l,k,n,b}} in \eqref{inter_exp} is \alert{treated as an optimization variable in \ac{ADMM}}
		\item Consensus variables are updated upon exchanging corresponding local \eqn{\zeta_{l,k,n,b}}'s among coordinating \acp{BS}
	\end{itemize}
\end{frame}

\subsection{KKT based Distributed Solution}

\begin{frame}{KKT based Distributed Solution}
	\begin{itemize}
		\item Decentralization methods involve significant signaling exchanges via backhaul
		\item \alert{Overhead is large for multi-antenna receivers} - iterative design should reduce the backlogged packets significantly in first few iterations
		\item To achieve that, we design precoders by solving the \ac{KKT} equations of the \acs{JSFRA} problem via \acs{MSE} reformulation
		\item \alert{Group update of all involved optimization variables} - to speed up the convergence of precoder design
	\end{itemize}
\end{frame}

\section{Simulation Results}

\subsection{Centralized Solutions}

\begin{frame}{Centralized Solutions}
\begin{figure}
	\centering
	\includegraphics[width=0.75\textwidth]{fig-2-5}
	\caption{System Model - \me{\lbrace N,N_B,K,N_T,N_R \rbrace = \lbrace 2,3,9,4,2\rbrace}}
\end{figure}
\end{frame}

\subsection{Distributed Solutions}

\begin{frame}{Distributed Solutions}
	\begin{figure}
		\centering
		\includegraphics[width=0.75\textwidth]{fig-3-2}
		\caption{System Model - \me{\lbrace N,N_B,K,N_T,N_R \rbrace = \lbrace 3,2,8,4,1 \rbrace}}
	\end{figure}
\end{frame}

\subsection{KKT based Approach}

\begin{frame}{Performance of KKT based Approach}
	\begin{figure}
		\centering
		\includegraphics[width=0.75\textwidth]{fig-9-3}
		\caption{System Model - \me{\lbrace N,N_B,K,N_T,N_R \rbrace = \lbrace 5,2,8,4,1 \rbrace}}
	\end{figure}
\end{frame}

\subsection{Time Correlated Fading Performance}

\begin{frame}{Time Correlated Fading Performance}
	\begin{figure}
		\centering
		\includegraphics[width=0.75\textwidth]{average_queue_over_time-3}
		\caption{System Model - \me{\lbrace N,N_B,K,N_R \rbrace = \lbrace 3,2,8,1 \rbrace} after \eqn{250} transmissions}
	\end{figure}
\end{frame}

\section{Conclusions}

\begin{frame}{Conclusions}
\begin{itemize}
\item We studied cross layer problem of designing transmit and receive beamformers based on the number of residual packets
\item Since the problem is nonconvex, we solve the problem iteratively by solving convex subproblems in each iteration
\item We also proposed a practical way of implementing the precoder design in a distributed manner by solving the \ac{KKT} expressions
\item Extensions of the proposed work in time-correlated fading scenario with limited number of information exchange cycle is in progress
\end{itemize}
\end{frame}


\begin{frame}
\begin{center}
{\color{blue}\Huge{Questions !}}
\end{center}
\end{frame}

\end{document}
